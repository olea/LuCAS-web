%%%%%%%%%%%%%%%%%%%%%%%%%%%%%%%%%%%%%%%%%%%%%%%%%%%%%%%%%%%
% Secci�n: Aplicaciones de texto vs aplicaciones gr�ficas %
%%%%%%%%%%%%%%%%%%%%%%%%%%%%%%%%%%%%%%%%%%%%%%%%%%%%%%%%%%%
\section{Aplicaciones de texto vs aplicaciones gr�ficas}

Linux no necesita un entorno de ventanas para funcionar. Ciertamente,
cuando comenz� no exist�a dicho entorno. La pantalla era un int�rprete
de comandos de aspecto similar a los Unix o una ventana de MS-DOS.
Las aplicaciones que funcionan bajo terminales o consolas las
llamaremos \emph{aplicaciones de texto}.

Tiempo despu�s se port� un sistema de ventanas llamado X/Window, muy
popular en el mundo Unix.

Es un sistema de control de mouse y pantalla, pero no maneja las
ventanas y operaciones con ventanas (como mover, minimizar, cerrar,
etc.). Por lo tanto hay que utilizar alg�n programa administrador de
ventanas. Se eligi� para el curso es el KDE. Existen muchos
otros entre los cuales est� el Gnome, CDE, WindowMaker y AfterStep.

Las aplicaciones que funcionan bajo X/Window las llamaremos
\emph{aplicaciones gr�ficas}. Necesitan X/Window para funcionar pero no
necesitan un administrador de ventanas. Sin embargo el administrador
de ventanas facilita el uso de los programas.

\subsection{Ventajas y desventajas}

Las aplicaciones de texto, tienen la ventaja de ocupar poco espacio,
ser r�pidas y la mayor�a tiene mucho desarrollo. Hay que pensar que
las terminales existen hace mucho tiempo.

Como desventajas se puede decir que los programas de texto son poco
amigables y tienen una interfaz restringida. Son ideales para tareas
administrativas de la computadora, terminales con enlaces lentos, y
software en general para computadoras de poca capacidad.

Como contrapartida existen las aplicaciones gr�ficas, con
una interfaz mejorada pero con mayor lentitud en mostrar informaci�n.
Son ideales para tareas de usuarios finales, personas con poca
pr�ctica en computaci�n, etc.

En este curso se van a usar aplicaciones gr�ficas en lo posible.

Las tareas administrativas m�s importantes (a�adir/eliminar usuarios,
configurar hardware, dar permisos, etc.) se pueden hacer en ambas
interfaces, tanto en la de texto como en la gr�fica. 











