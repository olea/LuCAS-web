%%%%%%%%%%%%%%%%%%%%%%%%%%%
% Secci�n: Bajada de Inet %
%%%%%%%%%%%%%%%%%%%%%%%%%%%
\section{Consiguiendo paquetes nuevos}

Es muy probable que los programas especializados no se encuentren en el
CD-ROM de GNU/Linux, al igual que las versiones actualizadas que se
deber�n copiar de otros lugares. Un buen lugar es Internet puesto
que normalmente las �ltimas versiones aparecen all�.

Los paquetes deben ser elaborados a partir de una cantidad de archivos
a instalar junto con datos de configuraci�n. Muchos programadores no
fabrican su Software en paquetes RPM. Por lo tanto tiene que haber
alguna persona dedicada a ``empaquetar'' esos archivos.

Esto quiere decir que, si bien existen versiones nuevas de algunos
programas, no necesariamente est�n en paquetes RPM.

\subsection{Saber sobre nombres de paquetes}

Para copiar paquetes primero hay que saber qu� es lo que estamos copiando.

Un nombre t�pico de paquete ser�a:
\begin{verbatim}
xmms-0.9.5-1.i386.rpm
\end{verbatim}

donde:

\begin{description}

 \item[{\tt xmms}] es el nombre del paquete.

 \item[{\tt 0.9.5}] es la versi�n, cuanto mayor, m�s nuevo es el
 	 paquete.

 \item[{\tt 1}] es la \emph{revisi�n}. Ser�a la versi�n del
	empaquetado, no del programa.

 \item[{\tt i386}] es la \emph{arquitectura}, o sea el tipo de m�quina
	 donde va a ejecutarse.

 \item[{\tt rpm}] es la extensi�n.

Ya veremos en detalle que hay que tener en cuenta al copiar.

\end{description}

\subsection{Buscando paquetes}

Se pueden buscar en un buscador com�n. Como puede ser
\sitio{www.google.com} o \sitio{www.altavista.com}. Si bien no es la
forma m�s �ptima, muchas veces es un buen comienzo.

Otra alternativa es ir a sitios especializados en el tema. Existen
varios \emph{rpmfinders}\footnote{Buscadores de paquetes RPM.}
para buscar infinidad de paquetes. Al principio suele deconcertar la
cantidad que existen. No s�lo para computadoras tipo PC sino para
diferentes arquitecturas.\footnote{Recordemos que Linux es un sistema
operativo multiplataforma.}

Un ejemplo de \emph{rpmfinder} es \sitio{http://www.rpmfind.net/linux/RPM}

Siempre nos debemos asegurar de encontrar los paquetes que digan {\tt
i386} para 386, 486 y Pentium I, {\tt i586} para Pentium MMX y K6 II/III
o {\tt i686} para Pentium II/III y similares.

%%%%%%%%%%%%
%%%% describir como copiar a la PC
Para bajar el paquete que hayamos seleccionado bastar� con presionar
el enlace correspondiente y se nos mostrar� el di�logo Guardar Como
de nuestro navegador. A la hora de elegir la ruta d�nde almacenar
nuestro paquete hemos de tener cuidado de no cambiar el nombre del
paquete puesto que, como hemos visto, este nos da informaci�n, sobre
versiones y tipos de arquitecturas.


\subsection{C�mo encontrarlo}

Una vez copiado al HOME, hay que ir a \menu{Archivo} \menu{Abrir} y deber�
aparecer el paquete a instalar.

Se�alando el paquete y posteriormente pulsando en \comando{Abrir} o con un
simple doble click, veremos la descripci�n junto con otros detalles
y un bot�n que indica \boton{instalar}. El procedimiento es el mismo que fue
descripto en la p�gina \pageref{kpackage-Instalacion}.



