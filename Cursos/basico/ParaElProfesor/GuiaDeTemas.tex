%%%%%%%%%%%%%%%%%%%%%%%%%%
% Seccion: Guia de Temas %
%%%%%%%%%%%%%%%%%%%%%%%%%%
\section{Gu\'ia de Temas}
% Para tener un indice y no perder el hilo de la clase

\subsection{Primera Clase}
\begin{itemize}

\item Proyecto GNU
	\begin{itemize}
	\item Historia de Stallman
	\item Niveles de Libertades
	\item Licencia GPL
	\item Ejemplos de GPL y no-GPL
	\end{itemize}

\item Historia de GNU/Linux
	\begin{itemize}
	\item Desarrollo del n�cleo
	\item Desarrollo de apps GNU
	\item Mezcla de ambos
	\item Linus Torvalds
	\end{itemize}

\item Comunidad
	\begin{itemize}
	\item Desarrollo de software libre
	\item Grupos de Usuarios de GNU/Linux
	\item LUGLi, LUGro, LUGar
	\item CVS
	\end{itemize}

\item Documentaci�n
	\begin{itemize}
	\item LDP
	\item LuCAS
	\item COMOs
	\item Listas de correo
	\item Sitios de inter�s
	\end{itemize}

\item Desarrollo de capas
	\begin{itemize}
	\item N�cleo (b�sico)
	\item X, Qt, KDE
	\item X, Gtk, GNOME
	\end{itemize}

\item Nociones de Multitarea y Multiusuario

\item Gr�fico versus Texto

\item Seguridad y virus

\item Nombres de archivo

\item Manejo de dispositivos

\item Atributos
	\begin{itemize}
	\item Archivos
	\item Directorios
	\end{itemize}

\end{itemize}


\newpage

\subsection{Segunda Clase}
\begin{itemize}

\item Repaso de dise�o por capas de X (espec�fico de KDE)

\item Uso de {\tt man pages} y \comando{Xman}

\item Configuraci�n del entorno KDE
	\begin{itemize}
	\item Diferencias entre el <<Centro de Control KDE>> y el <<Panel 
	de Control>> del Windows
	\item Configuraci�n del lenguaje
	\item Configuraci�n del fondo de Pantalla (con pruebas)
	\item Configuraci�n de Temas de Escritorio (con pruebas)
	\item Configuraci�n del tipo de teclado
	\item Configuraci�n de movimiento de ventanas con/sin contenido
	\item Localizaci�n de los botones de control de ventanas
	\end{itemize}

\item Aplicaciones b�sicas de KDE
	\begin{itemize}
	\item \comando{KFM} (KDE File Manager)
		\begin{itemize}
		\item Configuraci�n
		\item Modos de visualizaci�n de archivos
		\item Administraci�n de archivos b�sico (mover, copias, atributos, crear carpetas)
		\item Navegaci�n Web
		\item Configuraci�n de tipos MIME
		\end{itemize}
	\item \comando{kwrite} (Editor Avanzado, en castellano)
		\begin{itemize}
		\item Seteo de colores para programaci�n
		\item Utilizaci�n de marcadores
		\item Pruebas de visualizaci�n de diferentes tipos de archivo
		\end{itemize}
	\item \comando{ksnapshot} (Instant�nea, en castellano)
	\end{itemize}

\end{itemize}
