\section{Integraci�n de GPG con \comando{pine}} En la secci�n
\ref{seccion:gpg} se ha explicado a grandes rasgos los conceptos de la
herramienta de cifrado GPG y su uso gen�rico con archivos de cualquier
tipo. Quiz�s el m�todo descrito anteriormente no es el m�s c�modo para
usarlo con un cliente de correo, ya que al necesitar cifrar un mensaje
se deber�a hacerlo aparte en un archivo de texto, para luego
importarlo al cliente de correo; cuando se necesitan enviar varios
mensajes cifrados, se convierte en una tarea muy tediosa.

Pero el \comando{pine} tiene la posibilidad de configurarse para
tratar con mensajes cifrados o firmados\footnote{Los mensajes firmados
  no son mensajes ilegibles para todos excepto el destinatario, sino
  que cualquiera lo puede leer, y al tener una firma, se puede
  constatar que dicho mensaje lo ha escrito la persona quien dice
  ser.}, a continuaci�n se describir�n los pasos para configurarlo
correctamente:

\begin{enumerate}
\item Suponiendo que el archivo ejecutable \archivo{gpg} est� en
  \archivo{/usr/bin/gpg}, se debe ejecutar lo siguiente:

\begin{verbatim}
usuario@maquina:~$ ln -s /usr/bin/gpg ~/.gnupg/encrypt
usuario@maquina:~$ ln -s /usr/bin/gpg ~/.gnupg/gpgsign
\end{verbatim}

  Lo cual genera enlaces simb�licos del archivo ejecutable al
  directorio de instalaci�n de GPG en el directorio personal. Esto es
  para que \comando{pine} pueda distinguir el mismo ejecutable de las
  dos funciones: firmar y cifrar.
\item Ejecutar el \comando{pine} ingresando en la secci�n de
  configuraci�n del programa. Buscar la opci�n de configuraci�n
  denominada \comando{display-filters} e ingresar lo siguiente en ese
  campo:

\begin{verbatim}
_LEADING("-----BEGIN PGP ")_ /usr/bin/gpg
\end{verbatim}
  
  Esto le indica al \comando{pine} que ejecute GPG en caso de detectar
  la presencia de informaci�n cifrada o firmada en el cuerpo de un
  mensaje.
\item En el campo siguiente, el denominado \comando{sending-filters},
  se deben ingresar dos valores, que se incluyen a continuaci�n:
  \begin{itemize}
  \item 
\begin{verbatim}
~/.gnupg/encrypt -eastr _RECIPIENTS_
\end{verbatim}
  \item 
\begin{verbatim}
~/.gnupg/gpgsign -ast
\end{verbatim}
  \end{itemize}
  
  El primer valor indica a \comando{pine} que cifre (cuando se lo
  requiera el usuario) con el comando dado y a todos los receptores
  del mensaje en cuesti�n. Recordar que se debe poseer la clave
  p�blica de aquel al que queramos enviar un mensaje cifrado, de lo
  contrario el mensaje no se enviar�.
  
  El segundo valor indica a \comando{pine} que firme digitalmente el
  mensaje a enviar. GPG ser� ejecutado y pedir� la frase de contrase�a
  para realizar la firma, que se agregar� al final del mensaje.
\end{enumerate}

La figura \ref{fig:EncriptandoDesdeClientesDeCorreo-ConfDelPine}
muestra como debe quedar la configuraci�n de \comando{pine} para usar
GPG.

\figura{Configurando \comando{pine} para uso de GPG}{EncriptandoDesdeClientesDeCorreo-ConfDelPine}

