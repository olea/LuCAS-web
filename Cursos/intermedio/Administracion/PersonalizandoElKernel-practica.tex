\begin{practica}

\begin{ejercicio}
  \enunciado Compilar un n�cleo monol�tico y uno modular con las
  mismas opciones de configuraci�n, comparar los tama�os (tener en
  cuenta el tama�o de los m�dulos). Sacar conclusiones.

%  \solucion 
\end{ejercicio}

\begin{ejercicio}
  \enunciado Utilizar LILO para hacer m�ltiples entradas a ambos
  n�cleos, adem�s de los n�cleos que ya exist�an en el sistema.

%  \solucion
\end{ejercicio}

\begin{ejercicio}
  \enunciado Probar los n�cleos y verificar si existen errores al
  iniciar. Si da errores, ?`qu� tipo de errores da?. Si uno de ellos
  �nicamente tiene errores y ambos fueron configurados con las mismas
  opciones, ?`por qu� cree que da errores?.

%  \solucion
\end{ejercicio}

\begin{ejercicio}
  
  \enunciado Arrancar el sistema con ambos n�cleos y comparar los
  tama�os de �stos en memoria. ?`Cu�l de ellos tiene ventaja
  en la utilizaci�n de recursos?. ?`Y en el tiempo de ejecuci�n?.
  \begin{footnotesize}
    \textbf{Nota}: utilizar \comando{lsmod} para calcular el tama�o
    de los m�dulos en ejecuci�n y \comando{dmesg | grep Memory} para
    el tama�o del n�cleo
  \end{footnotesize}

%  \solucion 
\end{ejercicio}

\end{practica}


