%    Introducci�n al uso de GNU/Linux - Manual para el dictado de cursos
%    Copyright (C) 2000 Lucas Di Pentima, Nicol�s C�sar
%
%    This program is free software; you can redistribute it and/or modify
%    it under the terms of the GNU General Public License as published by
%    the Free Software Foundation; either version 2 of the License, or
%    (at your option) any later version.
%
%    This program is distributed in the hope that it will be useful,
%    but WITHOUT ANY WARRANTY; without even the implied warranty of
%    MERCHANTABILITY or FITNESS FOR A PARTICULAR PURPOSE.  See the
%    GNU General Public License for more details.
%
%    You should have received a copy of the GNU General Public License
%    along with this program; if not, write to the Free Software
%    Foundation, Inc., 59 Temple Place, Suite 330, Boston, MA  02111-1307  USA

%%%%%%%%%%%%%%%%%%%%%%%%%%%%%%
% Seccion: Dinamica de Clase %
%%%%%%%%%%%%%%%%%%%%%%%%%%%%%%
\section{Din\'amica de Clase}
% Para completar este documento, hay que anotar todas las preguntas hechas
% en clase.
% Publicar la lista de preguntas enviandola a los coordinadores de este 
% proyecto (ldipenti@ciudad.com.ar, ncesar@ciudad.com.ar)
%
% Esta seccion se encargara de recopilar aquellas preguntas que se les 
% puede hacer a los alumnos para dinamizar la clase un poco.

