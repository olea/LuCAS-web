\begin{practica}

\begin{ejercicio}
  \enunciado Hacer un script que compile un programa\footnote{Se puede
    realizar con \comando{gcc -c main.c}} y S�LO en el caso de que la
    compilaci�n sea exitosa, realice el enlazado (o \emph{linkado})
    del mismo\footnote{Aqu� podemos realizar \comando{gcc main.o -o
        main}.}.

%\solucion
\end{ejercicio}
 
\begin{ejercicio}
  \enunciado Suponiendo que el comando \comando{cant-mb-libres}
  retorna la cantidad de MB libres en el disco, hacer un script que
  compruebe la capacidad disponible, y si es mayor a 640MB, copia el
  directorio \archivo{/usr/local} al directorio \archivo{/backup}. En
  caso contrario mandar un correo a {\tt administrador@lejos.ch}, tambi�n
utilizando un hipot�tico comando: \comando{enviar-mail}.

%\solucion
\end{ejercicio}


\begin{ejercicio}
  \enunciado En base al listado del directorio \archivo{/backup},
  enviar a la impresora (o agregar al archivo \archivo{/dev/lp0} que
  es equivalente) los primeros 30 elementos. Si exist�a mayor cantidad
  de archivos escribir una l�nea final que diga ``y m�s...''.  Este
  ejercicio realizarlo con \comando{while} o \comando{for} en lugar de
  tuber�as.

%\solucion
\end{ejercicio}

\end{practica}