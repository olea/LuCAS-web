\section{Variables de entorno}

Las \emph{variables de entorno} tienen la funcionalidad de configurar
ciertos aspectos del entorno del int�rprete de comandos y otros
programas, que pueden cambiar con el tiempo. Estas variables se
establecen cuando se abre una sesi�n, y la mayor�a son configuradas
por los \emph{scripts} de inicio del int�rprete de
comandos\footnote{En el caso del int�rprete \comando{bash}, estos
  scripts incluyen el \archivo{.bashrc}, \archivo{.bash\_profile}, y
  otros.}.

Aunque se pueden establecer nombres de variables con min�sculas, por
costumbre se utilizan nombres en may�sculas, el comando para
establecer las variables de entorno se llama
\comando{export}\footnote{En \comando{bash}.}, y se utiliza de la
siguiente forma:

\begin{verbatim}
export VARIABLE=valor
\end{verbatim}

Para ver el contenido de una variable, se puede usar el comando
\comando{echo} de la siguiente manera:

\begin{verbatim}
echo $VARIABLE
\end{verbatim}
%$

Para eliminar una variable, se utiliza el comando interno del
int�rprete \comando{bash}, llamado \comando{unset} pas�ndole como
par�metro el nombre de la variable.

Es importante notar que una vez que se sale de una sesi�n, las
variables establecidas se pierden. Es por eso que si se necesita
disponer de variables espec�ficas cada vez que se abra una sesi�n en
GNU/Linux, es imprescindible agregar dichas configuraciones a los
archivos de inicio del int�rprete de comandos.

Otro uso com�n de estas variables es en los \emph{scripts}, programas
hechos en el lenguaje del int�rprete; las variables de entorno son de
gran ayuda para establecer configuraciones f�cilmente cambiables en
dichos programas.

