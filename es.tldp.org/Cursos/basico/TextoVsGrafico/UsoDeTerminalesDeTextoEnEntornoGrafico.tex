%%%%%%%%%%%%%%%%%%%%%%%%%%%%%%%%%%%%%%%%%%%%%%%%%%%%%%%%%%
% Secci�n: Uso de terminales de texto en entorno gr�fico %
%%%%%%%%%%%%%%%%%%%%%%%%%%%%%%%%%%%%%%%%%%%%%%%%%%%%%%%%%%
\section{Texto en entornos gr�ficos}

Hasta ahora hemos presentado a ambos tipos de aplicaciones como
opuestas y sin conexi�n entre ellas. Pero en la vida real no es
as�, es muy com�n usar aplicaciones de texto en entornos gr�ficos
Se utiliza un programa gr�fico que \emph{emula} (o \emph{simula} como
gusten) a una terminal de texto. Sobre esa terminal emulada se
ejecutan las aplicaciones de texto como puede ser
\comando{bash}, un int�rprete de comandos.

El \comando{bash} a su vez puede ejecutar otra aplicaci�n de texto.
Este es el funcionamiento normal. Cuando utilizamos el \emph{programa
gr�fico} \comando{konsole} este emula una terminal ejecutando
\comando{bash}. Y a partir del \emph{bash} tipeamos el nombre de otro
programa (por ejemplo \comando{mc}) para que se ejecute en esa
terminal. Para salir del \comando{mc} hay que presionar \boton{F10}.

Pero no s�lo se ejecutan aplicaciones de texto. Tambi�n se pueden
ejecutar aplicaciones gr�ficas que salgan en pantalla. Tranquilamente
estando en \comando{konsole} se puede tipear \comando{kedit} y saldr�
una ventana con el editor de texto.

\figura{ejecuci�n del \comando{kedit} en \comando{konsole}}{konsole-kedit}

Al probar esto �ltimo podemos observar que en \comando{konsole} no
aparece el \emph{prompt}\footnote{el prompt suele ser el s�mbolo {\tt \$} m�s
informaci�n del tipo {\tt [usuario@maquina directorio]}}. Existen dos
alternativas: cerrar el editor de texto o apretar \comando{Ctrl-C} para que
\comando{konsole} mate al editor. 

Existen otras alternativas pero no se discutir�n en este curso. 
