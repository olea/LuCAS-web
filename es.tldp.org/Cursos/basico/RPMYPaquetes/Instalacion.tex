%%%%%%%%%%%%%%%%%%%%%%%%
% Secci�n: Instalaci�n %
%%%%%%%%%%%%%%%%%%%%%%%%
\section{Instalaci�n}
\label{kpackage-Instalacion}

Uno de los prop�sitos que tiene el sistema de paquetes RPM es la
instalaci�n de paquetes nuevos. Se puede encontrar todo el conjunto de
paquetes a instalar en el CD-ROM de GNU/Linux.

Tambi�n se consiguen infinidad de paquetes en Internet listos para
bajar. Existen utilidades y sitios dedicados a encontrar paquetes de
un determinado tipo.

Por ejemplo si queremos utilizar el reproductor de mp3 llamado
 ``xmms'' y no se encuentra instalado:

\begin{enumerate}
 \item Insertamos el CD-ROM de GNU/Linux

 \item Si no se monta autom�ticamente, debemos ir a una terminal y
       montarlo escribiendo \comando{mount /mnt/cdrom}

 \item Ejecutamos el \comando{kpackage}, desde una Terminal o
       accediendo a trav�s de \menu{K-Utilidades-kpackage}

 \item Va a preguntar la clave del \usuario{root}, debido
       a que los paquetes se hacen disponibles a todos los usuario no
       tan s�lo a uno.

 \item Una vez en el programa \emph{kpackage} debemos ver una pantalla similar a
       la figura \ref{fig:kpackage-Inicial}
 \figura{Pantalla inicial del kpackage}{kpackage-Inicial}

 \item Buscamos el paquete ``xmms'', para eso vamos a \menu{Archivo} \menu{Buscar}
 Paquete. (fig. \ref{fig:kpackage-Busqueda}).
 \figura{Funci�n de b�squeda de paquetes}{kpackage-Busqueda}

 \item Si est� instalado va a aparecer con un icono de paquete que tiene
       una ``R'' (de RPM). En este caso, elija otro paquete que no est�
       intalado en la categor�a NEW.

 \item En cambio si no est� instalado, el icono es una N (de Nuevo) y
       se puede ver informaci�n sobre el
       paquete. (fig. \ref{fig:kpackage-Seleccionado-New-Xmms}).
 \figura{Vista de paquetes no instalados}{kpackage-Seleccionado-New-Xmms}

 \item El bot�n \boton{Examinar} contiene detalles acerca de la instalaci�n.
	(fig. \ref{fig:kpackage-Examinar-Xmms}).
 \figura{Detalles sobre la instalaci�n de paquetes}{kpackage-Examinar-Xmms}

 \item Para instalar el paquete s�lo hay que presionar el bot�n
       \boton{Instalar}.
\end{enumerate}


En el caso de que no haya problemas ya se podr�a ejecutar el \comando{xmms}
desde una terminal o consola como muestra la figura \ref{fig:konsole-xmms}.

\figura{Carga del xmms desde una consola}{konsole-xmms}

Para aquellos usuarios de la KDE 2.0 el proceso de instalaci�n es casi igual
pero existen algunos matices que se deben tener en cuenta. Para ejecutar
\emph{kpackage} podemos ejecutar \comando{kapackage} desde un terminal en
el que previamente hayamos ejecutado \comando{su} porque de otra manera
nos dejar� ver los paquetes instalados pero no permitir� la instalaci�n
de nuevos paquetes.
El entorno que nos presenta \emph{kpackage} en la versi�n de KDE 2.0 es
parecido al de la KDE 1.1.x con lo que se puede aplicar todo lo visto
hasta ahora sin ninguna compliaci�n.
\newpage
