%%%%%%%%%%%%%%%%%%%%%%%%%%%%%%%
% Secci�n: KDE - Introducci�n %
%%%%%%%%%%%%%%%%%%%%%%%%%%%%%%%


\section{Introducci�n}
El \emph{K Desktop Environment}, mejor conocido como KDE es uno de los
entornos de escritorio m�s conocidos y utilizados en la actualidad,
junto con el \emph{GNU Object Model Environment} o GNOME son quiz�s
los dos entornos mas utilizados y en los se est� realizando mayor
desarrollo.  La raz�n de su �xito radica en su facilidad de uso, su
similitud con el entorno de \emph{Microsoft Windows}, su funcionalidad
y su gratuidad.

?`En qu� radica un \emph{entorno de escritorio}?, quiz�s el lector, si
viene del ``mundo windows'', no se haya percatado de su existencia por
m�s que haya estado trabajando siempre con uno; un entorno de
escritorio consta de m�dulos de software trabajando en conjunto con el
\emph{servidor gr�fico llamado X} con el fin de proveer al usuario de
un �rea de trabajo (llamada \emph{escritorio}), una barra de acceso
r�pido a las aplicaciones y la posibilidad de usar \emph{�conos} y
\emph{carpetas} en el escritorio para organizar los archivos.

El KDE provee estas funciones, y varias m�s, como ser:
\begin{itemize}
\item Protector de pantalla (screensaver)
\item Bloqueo de sesi�n con contrase�a
\item Funciones \emph{drag\&drop}
\item Personalizaci�n de ventanas a trav�s de los \emph{Temas de Escritorio}
\item Uso de m�ltiples escritorios virtuales
\item Men�es personalizables
\end{itemize}

Para dar una mejor idea de como se ve un escritorio de KDE reci�n
instalado, se incluye la figura \ref{fig:EscritorioInicial}

\figura{Escritorio inicial del KDE}{EscritorioInicial}

